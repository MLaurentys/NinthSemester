
\documentclass[]{article}
\usepackage{amsmath}
\usepackage{bm}
\usepackage{amssymb}
\usepackage{listings}
\usepackage{mathtools}
\usepackage{tikz}

\DeclarePairedDelimiter\floor{\lfloor}{\rfloor}

\lstset{columns=fullflexible,
	mathescape=true,
	numbers=left,
	stepnumber=1,
	morekeywords={return, if, while, True, False},
	xleftmargin=5.0ex
}

\title{\vspace{-4.0cm}MAC0331 - Lista 1}
\author{Matheus T. de Laurentys, 9793714}

\begin{document}
	\maketitle
	\noindent
	\textbf{Q 1:} \\
	Algorithm details: if a point is intersecting a line segment, I will say that the point is in the region on the right of the segment.
	
	Both the segment insertion and the point look-up are based on a routine (1) that checks if a point is strictly to the left of the segment (the same used in class). Note that (1) is a $\textit{constant time}$ operation.
	
	The tree is going to be a self-balancing binary search tree, and the insertion algorithm uses $\textbf{any}$ point of the segment being inserted and (1) to decide if insertion is on the left/right of the segment it is compared to. Since it is a self-balancing binary tree, n consecutive insertions take time in $O(nlgn)$.
	
	In the same way, a single point can be checked against the segments on the binary tree. Since it is a balanced binary tree, the height is in $O(lgn)$, and, therefore, determining the region a point blongs to is also in $O(lgn)$ since (1) is a $\textit{constant time}$ operation.
	
	\textbf{Q 4:} \\
	This is exactly what I did in PE01. I will solve it with the sweep line method as described in class. Following is python code that solves the problem. Segment takes the polygon and returns a list of its segments. verify\_intersection returns whether the two segments intersect.
	
		\begin{lstlisting}
def treat_left(s, bst):
	bst.insert(s)
	ns = bst.get_neighbours(s)
	ret = verify_intersection(s, ns[0])
	if (ret) : return True
	ret = verify_intersection(s, ns[1])
	if (ret) : return True
	
def treat_right (s, bst):
	ret = False
	bst.remove(s)
	ns = bst.get_neighbours(s)
	if(ns[0] and ns[1]):
		ret = verify_intersection(ns[0], ns[1])
	return ret
def Scanline (segments):
	segments = segment(P1) + segment(P2)
	segments = sorted(segments, key=functools.cmp_to_key(compare_segments))
	heap, hmap = make_event_points(segments)
	bst = balanced_binary_tree(Node_Seg)
	while (not heap.empty()):
		pt = heap.get()
		for seg in pt.left:
			if(treat_left(segments[seg], bst)):
				return True
		for seg in pt.right:
			if(treat_right(segments[seg], bst)):
				return True
	return False
	\end{lstlisting}
	
	$\textbf{Q5:}$\\
	This is another application of the same sweep line algorithm above. I will make a change on the verify\_intersection(), segment() methods. The only line cheanged is line 17.
	
	Now, verify\_intersection() will check $\Delta x^2 + \Delta y^2 < \Delta r^2$ to return whether there is an intesection. segment() will, instead of getting all the edges of the polygon, generate a single segment given by $[(x-r,y), (x+r,y)]$. Therefore, line 17 will be:\\
	 $\textbf{17.}$ segments = [segment(D) for D in disks]
	 
	 This algorithm is in $O(nlgn)$ because sorting the disks is in that class, and there is a linear amount (2n) of event points generated (each event is in $O(lgn)$).
\end{document}