
\documentclass[]{article}
\usepackage{amsmath}
\usepackage{bm}
\usepackage{float}
\usepackage{amssymb}
\usepackage{listings}
\usepackage{mathtools}
\usepackage{tikz}

\DeclarePairedDelimiter\floor{\lfloor}{\rfloor}

\lstset{columns=fullflexible,
	mathescape=true,
	numbers=left,
	stepnumber=1,
	morekeywords={return, if, while, True, False},
	xleftmargin=5.0ex
}

\title{\vspace{-4.0cm}MAC0331 - Lista 6}
\author{Matheus T. de Laurentys, 9793714}

\begin{document}
	\maketitle
	\noindent
	\textbf{Q 4:} \\
	The Voronoy diagram of a regular polygon with $n$ vertices consists of a vertex $v$ in the circumcenter of the polygon and $n$ rays that start in $v$, cross the midpoint of each edge, on towards infinity. The Delaunay graph is equal to the original regular polygon.
		\noindent
	\textbf{Q 6:} \\
	If $P$ consists of $n-1$ co-linear points, with adequate distance, and a point $v$ non co-linear, then the Delaunay diagram of $P$ is such that $d(v)=n-1$. \\
	Proof by induction: \\
	$\cdot$ Any two points in a line will be such that $d(v)=2$. This is simple if you consider that $v$ and $p_i$ will split a semi-plane in the Voronoi diagram.\\
	$\cdot$ Consider a collection C of $k$ co-linear points such that the Voronoy diagram of $C + \{v\}$ is such that $d(v) = k$.
	$\cdot$ Take $u$ the most distant point in $C$ to $v$ and $z$ any other point. We can add $w$, more distant to $v$ than $u$ in the direction $\overrightarrow{zu}$ in a way that we keep the initial property.
	
	
\end{document}