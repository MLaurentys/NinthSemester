\documentclass[]{article}
\usepackage{amsmath}
\usepackage{bm}
\usepackage{amssymb}
\usepackage{listings}
\usepackage{mathtools}

\DeclarePairedDelimiter\floor{\lfloor}{\rfloor}

\lstset{columns=fullflexible,
	mathescape=true,
	numbers=left,
	stepnumber=1,
	literate=
	{=}{$\leftarrow{}$}{1},
	morekeywords={se,entao,para,devolva,continua},
	xleftmargin=5.0ex
}

\title{\vspace{-4.0cm}MAC0331 - Lista 1}
\author{Matheus T. de Laurentys, 9793714}

\begin{document}
	\maketitle
	\noindent
	\textbf{Q 2:} O problema com a ideia é que manter os pontos da linha separadora todos no mesmo conjunto (esquerda ou direita) pode estragar a divisão dos pontos. Por exemplo, se fossemos dividir a instância [(0,0), (1, 0), (1, 2), (1,3),(1,4)] normalmente, obteriamos E=[(0,0), (1, 0), (1, 2)] e D=[(1,3),(1,4)] e, com a proposta do professor, obeteriamos E=[(0,0), (1, 0), (1, 2), (1,3),(1,4)] e D=[].\\
	\textbf{Q 4:} 
	\begin{lstlisting}
	Combine (X, Y, a, p, r, $d_E$, $d_D$)
	d = min{$d_E, d_D$}
	q = $\floor{(p+r)/2}$
	(f, t) = Candidatos (X, a, p, r, d)
	// so vai comparar pontos da esquerda com pontos da direita
	para i = 1 ate t - 1 faca
		se f[i] > q:
			entao continua
		para j = i + 1 ate min{i + 7, t} faca
			se f[j] $\le$ q:
				entao continua
			d' = Dist(X[f[i]], Y[f[i]], X[f[j]], Y[f[j]])
				se d' < d
					entao d = d'
	devolva d$$
	\end{lstlisting}
	O algoritmo acima continua linear. A prova e a mesma da feita sem as condicionais das linha 7 e 10, vista em aula.
\end{document}